\documentclass[12pt]{article}
\usepackage{graphicx}
\usepackage{amsfonts }
\usepackage{amsmath}
\usepackage{fullpage}
\usepackage{amssymb}
\usepackage{amsthm}


\theoremstyle{definition}
\newtheorem{teo}{Theorem}[section]
\newtheorem{cor}[teo]{Corollarie}
\newtheorem{lem}[teo]{Lemma}
\newtheorem{conj}[teo]{Conjecture}
\newtheorem{prop}[teo]{Proposition}
\newtheorem{defi}[teo]{Definition}
\newtheorem{prob}[teo]{Problem}

\newcommand{\inte}{\text{int}}
\newcommand{\skel}{\text{Skel}}
\begin{document}
\title{Billera-Lee polytopes, $r$-stacked polytopes and regular triangulations}
\author{Jos\'e Alejandro Samper}
\maketitle
\section{Introduction}
The goal of this project is to produce examples to support help us support a positive answer to a question by S. Murai and E. Nevo. In short  $P$ is a simplicial polytope with $g_r= 0$, where $(g_0,g_1, \dots g_[d/2])$ is the simplicial $g$-vector, then the stacked triangulation of P (the they showed exists) is regular, that is, it is the projection of the lower pointing faces of a polytope $Q$ with $\dim Q = \dim P + 1$ that projects to $P$. 

Regular triangulations are very nice and have been widely studied as a combinatorial structure that comes with a polytope. They always exist and can be parametrized as the vertices of a very nice polytope called the {\bf Secondary polytope} of $P$ or  the {\bf GKZ polytope} of $P$. Their combinatorial structure is fascinating and understanding it reveals a lot of information about the polytope. 

\section{The $g$-vector and the $g$-theorem}
The stacked triangulation of a polytope with $g_2 = 0$ is easily shown to be regular. The key behind the proof is that 
this polytopes are easily constructed inductively. On the other hand, polytopes with those characteristics do not arise
naturally in the study of geometry, but are interesting for theoretical reasons. 

The only big family of polytopes I know that has $g_r=0$ is the family of Billera-Lee polytopes. These are polytopes that 
were constructed to classify the family of all face numbers of simplicial polytopes. To construct them, one choose a 
suitable g-vector (the entries of this vector satisfy some prescribed inequalites) and constructs this polytope as the
boundary of a line shelling subcomplex of a special cyclic polytope of one dimension higher than the desired dimension. 

Now that we have this polytopes we can produce the stacked triangulation using a simple combinatorial rule (described by 
(see Nevo-Murai reference) 
\end{document}